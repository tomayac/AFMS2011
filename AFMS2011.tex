\documentclass[runningheads,a4paper, twocolumn]{llncs}

%%%%%%%%%% preamble %%%%%%%%%%

% margins
\usepackage[left=1.8cm, right=1.8cm, top=2.54cm, bottom=2.54cm]{geometry}

% autoref command
\usepackage[pdftex,urlcolor=black,colorlinks=true,linkcolor=black,citecolor=black]{hyperref}
\def\sectionautorefname{Section}
\def\subsectionautorefname{Subsection}
\def\figureautorefname{Fig.}
\def\subfigureautorefname{Fig.}

% listings and Verbatim environment
\usepackage{fancyvrb}
\RecustomVerbatimCommand{\Verb}{Verb}{fontsize=\footnotesize}
\RecustomVerbatimEnvironment{Verbatim}{Verbatim}{fontsize=\fontsize{7.5pt}{9pt}}
\usepackage{listings}
\lstloadlanguages{SPARQL} 
\lstdefinelanguage{RDF}{morekeywords=@prefix}
\lstset{frame=bottomline,captionpos=b,numberbychapter=false,
        aboveskip=0em,belowskip=.5em,abovecaptionskip=.3em,belowcaptionskip=.6em,
        basicstyle=\ttfamily\fontsize{7pt}{8.5pt}\selectfont,stringstyle=\em,showstringspaces=false}

% abstract keywords
\newcommand{\keywords}[1]{\par\addvspace\baselineskip
\noindent\keywordname\enspace\ignorespaces#1}

% todo macro
\usepackage{color}
\newcommand{\todo}[1]{\noindent\textcolor{red}{{\bf \{TODO} #1{\bf \}}}}

% references
\usepackage{cite}

% inclusion of graphics and PDF documents
\usepackage[pdftex]{graphicx}
\usepackage[final]{pdfpages}
\usepackage[bf,small]{subfigure}
\setlength{\floatsep}{1em}
\setlength{\dblfloatsep}{1em}
\setlength{\textfloatsep}{1em}
\setlength{\dbltextfloatsep}{1em}
\setlength{\belowcaptionskip}{-1em}

% better typography
\usepackage[tracking, spacing, kerning]{microtype}

% drawings
\usepackage{tikz}
\usepackage{tkz-graph}
\usetikzlibrary{matrix,arrows,decorations.pathmorphing,shapes}

\newcommand{\bulletnumber}[1]{%
   \tikz[baseline=-2.5]%
   \node[circle,text=white,fill=gray,anchor=west,inner sep=1pt] {\rm #1};%
}

% algorithm typesetting
\usepackage[small,bf]{caption}
\usepackage{algorithmic}
\usepackage{float}
\newfloat{algorithm}{tpbh}{}
\floatname{algorithm}{Algorithm}
\def\algorithmautorefname{Algorithm}
\newcommand{\algorithmtop}{\rule{\columnwidth}{.4pt}\vspace{-1.3em}}
\newcommand{\algorithmbottom}{\vspace{-0.6em}\rule{\columnwidth}{.4pt}\vspace{-0.6em}}

% equations typesetting
\usepackage{amsmath}
% add parenthesis around equation references
\makeatletter
\def\tagform@#1{\maketag@@@{\ignorespaces#1\unskip\@@italiccorr}}
\let\orgtheequation\theequation
\def\theequation{(\orgtheequation)}
\makeatother

% abstract
\renewenvironment{abstract}{\section*{\makebox[3.5cm]{}Abstract}}

% lay-out
\usepackage{needspace}

% conditionals
\usepackage{ifthen}
\provideboolean{showauthors}
\setboolean{showauthors}{true}


% make sure that OWL-S does not get hyphenated
\newcommand{\owls}{\mbox{OWL-S}}


%%%%%%%%%% document %%%%%%%%%%

\begin{document}
% For every picture that defines or uses external nodes, you'll have to
% apply the 'remember picture' style. To avoid some typing, we'll apply
% the style to all pictures.
\tikzstyle{every picture}+=[remember picture]

\mainmatter


%%%%%%%%%% title %%%%%%%%%%

\title{Description and Interaction of RESTful Multimedia Services for~Automatic Discovery and Execution}
\ifthenelse{\boolean{showauthors}}
{\author{Ruben Verborgh$^{1}$ \and Thomas Steiner$^{2}$ \and Rik Van de Walle$^{1}$ \and Joaquim Gabarr\'{o} Vall\'{e}s$^{2}$}
\institute{Ghent University -- IBBT, ELIS -- Multimedia Lab\\Gaston Crommenlaan 8 bus 201, B-9050 Ledeberg-Ghent, Belgium\\
\urldef{\emails}\path|{ruben.verborgh,rik.vandewalle}@ugent.be|\emails\\
\url{http://multimedialab.elis.ugent.be/}
\and Universitat Polit\'ecnica de Catalunya -- Department LSI\\
08034 Barcelona, Spain\\
\urldef{\emails}\path|tsteiner@lsi.upc.edu|\emails\\
\url{http://www.lsi.upc.edu/}
}}
{\author{\institute{}\vspace{3.5cm}}}

\twocolumn[
  \begin{@twocolumnfalse}
    \maketitle
  \end{@twocolumnfalse}
  ]


% use Courier from this point onward (used CM for author e-mails)
\renewcommand{\ttdefault}{pcr}


%%%%%%%%%% abstract %%%%%%%%%%

\begin{abstract}
%%%% Responsible: THOMAS %%%%
\todo{
Lorem ipsum dolor sit amet, consectetur adipiscing elit. Suspendisse gravida fringilla ipsum nec tincidunt. Curabitur fermentum tincidunt semper. Suspendisse at orci quam. Ut risus est, mollis nec placerat et, lacinia sed ligula. Nunc varius felis vitae elit mollis laoreet. Nulla congue, sem id ullamcorper convallis, nibh massa scelerisque elit, at congue purus eros vel metus. Nulla purus turpis, egestas eget vehicula et, blandit in nibh. Mauris id sapien velit, in viverra neque. In hac habitasse platea dictumst. Phasellus at ante vitae lectus scelerisque gravida. Curabitur bibendum, libero non interdum rhoncus, diam ante volutpat lorem, a volutpat ante mauris vel nisi. Donec nisi felis, rhoncus et suscipit sit amet, porta eget felis. Maecenas aliquam interdum elit, sed fringilla lorem fringilla non.
In hac habitasse platea dictumst. Sed blandit diam sed sem dignissim in aliquet massa consequat. Integer aliquam congue justo a rhoncus. Sed sit amet interdum eros. Nunc aliquam, nisl vitae pulvinar posuere, odio justo malesuada nunc, non sodales tellus diam quis purus. Aenean sed neque quis dui ultricies cursus a vitae orci. Nulla facilisi. Nulla lacus nisl, cursus at vehicula ac, malesuada non est.}
\end{abstract}


%%%%%%%%%% body %%%%%%%%%%

\section{Introduction}
%%%% Responsible: RUBEN %%%%
The immense diversity of various multimedia analysis and processing algorithms makes it difficult to integrate them in an automated platform to perform compound tasks. Yet, recent research has indicated the importance of the fusion of different techniques~\cite{Atrey:2010p3072}. But how can different algorithms interoperate, if there are no agreements or guidelines on how communication should happen? And how can a coordinating platform select algorithms based on their capabilities, in lack of a formal description detailing their preconditions and postconditions?

In this paper, we show how to lift multimedia algorithms to the level of Semantic Web services with a formal description mechanism that follows a pragmatic approach. Rather than reinventing the existing methodologies, which focus on low-level details, we want to express an algorithm's functionality in a way that captures its functionality without requiring lengthy specifications. Our intention is to use existing standards such as the HTTP protocol, Link headers, and URI templates and reuse common best practices for implementing multimedia algorithms as true Semantic Web services. The aim is a versatile description and communication model, enabling fully automated service discovery and execution, even under changing conditions. The sole starting point is a Web address of a server---all other information is gathered at runtime.

We will explain our approach by three real-world multimedia use cases, each of which represents challenges that are currently not fully addressed by alternative techniques. They will illustrate the power of our method and hopefully convince the reader that its apparent simplicity accommodates far-reaching possibilities.

The remainder of this paper is structured as follows: \todo{Describe remainder of the paper}. As a remark, we use the terms ``API" and ``(Web) service" synonymously.

We have implemented a sample multimedia Web service with mock data that follows our description approach. It can be tested at \url{http://restdesc.no.de/}.

\section{Related Work}
%%%% Responsible: THOMAS %%%%
\subsection{Web Service Description Language}
The description of Web services has a long history. The XML-based Web Service Description Language~(WSDL, \cite{WSDL1, WSDL2}) provided one of the first models. WSDL focuses on the communicational aspect of Web services, looking from a message-oriented point of view. The details of the message format are written down in a very verbose way and concretized to actual bindings such as SOAP~\cite{SOAP} or plain HTTP~\cite{HTTP}. Finally, the description can contain endpoints which implement the specified bindings.

For our use case, we spot two major problems with the use of WSDL. First, WSDL only provides the mechanisms to characterize the technical implementation of Web services. It does not provide the means to capture the functionality of a service. For example, a service that counts the number of words in a text will be described by WSDL as an interface which accepts a string and outputs an integer. Clearly, an infinite number of algorithms share those input and output properties, so this information is insufficient to infer any meaning or functionality.

Secondly, in practice, a WSDL description is used to generate module source code automatically, which is then compiled into a larger program. If the description changes, the program no longer works, even if such a change leaves the functionality intact. A concrete example of such brittleness is the switch from Int32 to Int64 IDs that at some point occurred Google's AdWords API\footnote{\url{http://www.somebits.com/weblog/tech/bad/whySoapSucks.html}}, a small change in the service's WSDL file that required the complete recompilation of the relevant pieces of source code. So WSDL is not well adapted to real-world circumstantial changes.

The above problems indicate why WSDL cannot offer automatic service discovery at runtime and why we should investigate other possibilities.

\subsection{Semantic Annotations for WSDL}
The W3C Recommendation named Semantic Annotations for WSDL and XML Schema~(SAWSDL, \cite{SAWSDL}) describes a way how to add semantic annotations to various parts of a WSDL document such as interfaces and operations, and input and output message structures. In addition to that, Web services can be assigned a category, with the objective of making them discoverable in a central registry of Web services. SAWSDL also defines an annotation mechanism for specifying the data mapping of XML Schema~(\cite{xmlschema11-1}, \cite{xmlschema11-2}) types to and from ontologies, often referred to as up- and down-lifting. While the standard fulfills parts of our requirements, it inherits all the disadvantages from WSDL, specifically its brittleness and verbosity.

\subsection{Web Application Description Language}
The Web Application Description Language~(WADL, \cite{WADL}), which is also XML-based, is another Web service description format, which advocates proper use of all the aspects of the HTTP protocol, instead of its degradation to a tunneling mechanism as in SOAP. Services that behave in this way are oftentimes (incorrectly\footnote{We prefer the term ``HTTP interface", where most API providers use the buzzword ``RESTful API".}) called RESTful~\cite{FieldingREST}, the properties of which we will explain in \autoref{RESTfulServices}. While WSDL~2.0 is also capable of specifying bindings to RESTful endpoints, it still requires the abstractions that enable bindings to SOAP and others. WADL, on the other hand, was tailored to the needs of RESTful services, but only exists as a W3C Member Submission and will most likely never reach the W3C Recommendation status of WSDL~2.0~\cite{WADLTeamComment}.

In addition to that, WADL still suffers from the same problem: it does emphasize the technical properties of the underlying service and does not leave any room for the semantics of the task it performs. This also means that there is no way to automatically discover services based on the desired functionality. Therefore, there is no reason why WADL would be used any differently than WSDL, as also argued by Joe Gregorio in~\cite{GregorioWADL}.

The main criticism by the REST community, however, is that WADL documents beforehand, what according to the REST principles~\cite{FieldingREST} should be discovered dynamically at run-time. One of the fundamental properties of REST is the so-called hypermedia constraint, which basically can be summarized as the constraint that each server response contains the possible next steps that the (stateless) client can take next. It is to be noted that WADL could be used in this way at run-time, however, most usage today is beforehand.

\todo{However, I would think that WADL can be interesting when a certain amount of detail is required.\\
So why don't we output add a \Verb!Link! header to a WADL description (or WSDL, or whatever, just to be generic) on HTTP \Verb!OPTIONS!?\\
Could it be possible {\em in theory} to generate a WADL description from RESTdesc (given the description and the use of RESTful practices)?}

\subsection{Semantic Markup for Web Services}

\owls~\cite{OWLS} is a an OWL~\cite{OWL} ontology for describing Semantic Web services in RDF~\cite{RDF}. A service description consists of three parts: a profile, a model and a grounding. Some aspects of profile and model are very similar, in the sense that they both describe input, output, preconditions and effects. The difference is that the profile is used for discovery, while the model is used to control the interaction.

Here, for the first time, we have an actual focus on the functionality of a service which is separate from how the interaction happens. However, whether this separation was successful is debatable, since there is no way to enforce the consistency of profile and model of a single service. Finally, the grounding part specifies the implementation of the service. The \owls\ submission defines a grounding to WSDL, but other groundings are possible (e.g.,~\cite{Verborgh:2010p2746}). This means that the \owls\ description describes the functionality, whereas its grounding describes the communication.

At least, this is what it is supposed to do. \owls\ input and output types provide more or less the equivalent of what a WSDL message format contains, albeit with RDF types, so there is only a minimal added semantic value on that level. The real possibilities lie in the use of preconditions and postconditions (the latter under the form of result effects), which allow to express complex relationships between input and output values, finally capturing the semantics and functionality of the service.

Unfortunately, there is no obligation to use these conditions and the \owls\ submission only mentions the rule languages KIF~\cite{KIF}, DRS~\cite{DRS} and SWRL~\cite{SWRL}, in order of increasing verbosity. Extensions to more light-weight rule languages, such as Notation3 Logic~\cite{N3Logic}, are possible~\cite{Verborgh:2010p2746}. The real problem here is that that none of those languages are integrated into the main service description, but rather form subdocuments within it, which require a separate interpretation. The conditions thus live in another context, which are solely linked by identifiers but not by semantics. As a result, agents lacking support for the used rule language could parse the \owls\ constructs in a service description document and skip the parts they fail to understand---effectively ignoring important conditions they should reckon with\footnote{This behavior resembles that of Web browsers without JavaScript support: they parse and render HTML but ignore any \Verb!script! tags. Service descriptions, in contrast, {\em always} require a full interpretation for correct functionality.}.

Furthermore, while \owls\ offers functional descriptions capable of automatic discovery of the capabilities of a single service, it does not provide mechanisms to express its relation to other services. Also, descriptions contain redundancies and require a fair amount of vocabulary, even to express conceptually simple services, and rely on external groundings for technical implementations.

\subsection{\todo{The SPARQL approach}}
\todo{The SPARQL approach, a.k.a. forcing services into behaving like a SPARQL endpoint. See Linked Open Services and my previous paper.}

\subsection{Universal Description Discovery and Integration} \label{sec:uddi}
The XML-based OASIS standard Universal Description Discovery \& Integration (UDDI,~\cite{UDDI}) was developed to enable the definition of a set of services supporting the discovery and description of (i) businesses, organizations, and other Web service providers, (ii) the Web services that those institutions offer, and finally (iii) the technical interfaces, which may be used to access those services. UDDI is based on a (at the time of the specification) common set of industry standards, including HTTP, XML, XML Schema, and SOAP. The standard was designed to allow for description and discovery of both, public services, and non-public in-house services. It was meant to be used as a service broker where parties interested in a special service (like, for example, shipping address verification) could go to and retrieve a list of service providers offering the desired service. Such services would be described in the so-called Green Pages, including not only technical details, but also contact details of the Web service provider.

While for various reasons out of scope of this paper UDDI could not gain the adoption its creators had hoped for, the overall idea of automatically being able to select a service from a (not necessarily central) registry of services still seems useful to us. We will show in Section~\ref{sec:restdesc-discovery} how we imagine this idea to work decentralized and dynamically using our approach.

\section{RESTful Multimedia Services}
\label{RESTfulServices}
%%%% Responsible: THOMAS %%%%
When we say RESTful service invocation we refer to the following REST principles~\cite{FieldingREST}:
\begin{itemize}
\item Servers and clients are separated from each other by a uniform interface. Both servers and clients have well-defined responsibilities, also referred to as separation of concerns. This is to guarantee maximum independence from the one and the other.
\item All client requests are stateless, this means that each request from a client has all the information that the server needs to process it.
\item Responses must define themselves as cacheable or not using standard HTTP caching techniques.
\item When layered systems (like load-balancing) are used, this fact must not be exposed to the API user.
\end{itemize}

In RESTful APIs, resources are identified by URIs. A resource is to be differentiated from its representation. For example, a set of RDF triples (the resource) might be represented in different serializations (syntaxes), such as RDF/XML or Turtle. When of these representations gets manipulated, there is enough information to manipulate the represented resource, always given the permission to manipulate it. Messages need to be self-descriptive, that is, for example the media type of a message needs to make clear what can be done with this message. Each representation needs to communicate relevant related resources, or next steps the client can take at each state.

To make this clearer, we introduce two related services, one for face detection, and the other for face recognition. A user agent can upload a photo to the face detection service and use it to check for the existence of faces in the uploaded image. If faces are found, the user agent can use the face detection service to try to find out more details on the persons whose faces are contained in the image. Each image is considered a resource, for example represented by a binary image file (like \url{/photos/1}). Each potential face is a resource represented for example by an RDF document serialized in Turtle, or a cropped version of the entire image showing only the particular face (like \url{/photos/1/faces/1}). Each person is a resource represented for example by a string of the person's name (like \url{/photos/1/persons/1}). Some of the potential next steps are, for example, after detecting faces, a link to a service that allows for recognizing these faces, or starting from the first person on an image, a link to the next person on the image.

\section{RESTdesc Semantic Description}
\label{RESTdesc}
%%%% Responsible: RUBEN %%%%
\subsection{Motivation}
The answer to the question whether Web service description and Web service discovery are necessary needs to be split up in two parts.

On the one hand there is the question whether Web service description is needed. In RESTful systems, the common opinion is that each message should be self-descriptive enough so that user agents can make sense of each message, given a documented media type that the message is serialized in. On a pure technical layer this works well. For example, let us imagine that the response from an ``I'm feeling lucky"\footnote{\url{http://www.google.com/support/websearch/bin/answer.py?&answer=30735}}-type image search API might be a single image of the media type \texttt{image/gif}. This gives the user agent enough information to process the response with its  Graphics Interchange Format (GIF) library, however, a priori it is not clear that the image stands in relation to a search query that the user agent has used as an input. While we could perfectly use OpenSearch~\cite{OpenSearch} to describe this API, only a slight variation of the API renders using OpenSearch impossible. Therefore, let us imagine a Web font preview API, where you give the name of a Web font as an input, and get a GIF image with a preview of the text ``The quick brown fox jumps over the lazy dog" in that very Web font as an output. There is currently no way to describe such API, and yet it might be desirable for a Web font vendor to announce the availability of it. 

The second question is whether automatic discovery of Web services is needed. The first approach for automatic service discovery was UDDI, outlined in Section~\ref{sec:uddi}. It was driven by the vision that central service registries would serve as so-called Green Pages for parties interested in a specific service. The problem with this approach, however, is that companies do not work this way. There is always a human being involved in the process\footnote{\url{http://soa.sys-con.com/node/164624}}. We see the potential of service discovery in the generation and run-time supervision of automatic execution plans as outlined in~\cite{Verborgh:2011p4792}, which can highly profit from discoverable service descriptions. The before-mentioned OpenSearch protocol allows for an interesting use case of service discovery. Web pages can reference an OpenSearch description that user agents like in the concrete case the Google Chrome browser can process and offer site-specific search automatically\footnote{\url{http://www.google.com/chrome/intl/en/webmasters-faq.html\#osdd}}. If we adapt this idea to our approach, user agents could dynamically offer services related to a current resource, if the resource points to a service description.
% You could make me and my promotor happy by \cite{Verborgh:2011p4792} as an example of automated use of services ;)

\subsection{Deriving a functional description}
By now, it is clear that we aim to provide a semantic method to express the functionality of a service---as well as its communication---in a concise way that appeals to humans and can be processed automatically. The word ``semantic'' obviously hints at the Semantic Web~\cite{SemanticWeb} and its core language RDF~\cite{RDF}.

Let us first revise what we actually want to express. Continuing the example of \autoref{RESTfulServices}, an informal expression for photo retrieval could be:
\begin{multline}\label{eq:Informal}
    \textit{I can retrieve a photo by going to \Verb!/photos/!}\\[-.3em]
    \textit{ and appending its identifier.}
\end{multline}

\noindent An intuitive formalization of the above would be
\begin{multline}\label{eq:NaiveFormal}
    \makebox[5.2cm]{$hasUri(request, \{\mbox{``\!\Verb!/photos/!''}, id\})\, \land$} \\[-.2em]
    hasResponse(request, resp) \land represents(resp, photo)\\[-.2em]
    \land photoId(photo, id)
\end{multline}
which is straightforward to represent in RDF:
\begin{multline}\label{eq:NaiveRdf}
    \fontsize{8}{10}\texttt{:request :uri ("/photos/", :id);} \\[-.5em]
    \fontsize{8}{10}\texttt{:response [ :represents [:photoId :id] ].}
\end{multline}
Upon closer inspection, it is clear that the formalization~\ref{eq:NaiveFormal}---and thus its RDF counterpart~\ref{eq:NaiveRdf}---does not contain all the semantics of the informal expression~\ref{eq:Informal}. While~\ref{eq:Informal} implies~\ref{eq:NaiveFormal}, the~opposite implication \mbox{$\ref{eq:NaiveFormal} \Rightarrow \ref{eq:Informal}$} is broken, and thus the equivalence does not hold. Indeed, fragment \ref{eq:NaiveRdf} states that there exists \emph{some} request which returns the photograph with the identifier specified in its URI. It does however not convey the implicit intention of~\ref{eq:Informal} that \emph{all} requests with this URI structure behave the same way. This is a problem of existential~$\exists$ versus universal~$\forall$ quantification, which has important consequences and should be dealt with formally.

Revising~\ref{eq:QuantifiedFormal} with quantifiers gives:
\begin{multline}\label{eq:QuantifiedFormal}
    \makebox[4.4cm]{$\forall\, photo : \exists\; id, request, uri, resp :$} \\[-.2em]
    \shoveleft{hasUri(request, \{\mbox{``\!\Verb!/photos/!''}, id\})\, \land} \\[-.2em]
    \shoveleft{hasResponse(request, resp) \,\land\, represents(resp, photo)}\\[-.2em]
    \land photoId(photo, id)
\end{multline}

\noindent However, this still remains insufficient, because the universal quantification introduces the claim that \emph{every} photograph in the world possesses an identifier---a false statement for the majority of photographs, with the exception of those uploaded to the server. Similarly, requests exist for such photographs only. Looking back at the informal expression~\ref{eq:Informal}, we now spot the (again, implicit) assumption that the photograph we want to retrieve has a~known identifier.

Therefore, our last revision of the formal expression takes into account this notion as follows:
\begin{multline}\label{eq:Formal}
    \makebox[7.8cm]{$\forall\, photo, id : photoId(photo, id) \Rightarrow
                    \exists\, request, uri, resp :$}\\[-.2em]
    \shoveleft{hasUri(request, \{\mbox{``\!\Verb!/photos/!''}, id\})} \\[-.2em]
    \shoveleft{\land hasResponse(request, resp)} \\[-.2em]
    \land\, represents(resp, photo)\makebox[3.38cm]{}
\end{multline}

The above expression now corresponds to the intended meaning of~\ref{eq:Informal}: that a representation of every photograph with an identifier can be retrieved by following the constructed URI. Now the issue of expressing~\ref{eq:Formal} in RDF remains. The original RDF specification~\cite{RDF} does not include a form of quantifiers. Although attempts have been made in the past (e.g., \cite{RDFQuantifiers}), the most successful is Tim Berners-Lee's Notation3~\cite{Notation3}, which as an added benefit also includes syntactical support for implications.

Expressing~\ref{eq:Formal} in Notation3 gives:
\begin{multline}\label{eq:FullNotation3}
    \makebox[2.7cm]{\fontsize{8}{10}\texttt{@forAll :photo, :id.}}\\[-.5em]
    \fontsize{8}{10}\texttt{@forSome :request.}\makebox[5.47cm]{}\\[-.5em]
    \fontsize{8}{10}\texttt{\{:photo :photoId :id.\}}\makebox[4.37cm]{}\\[-.5em]
    \fontsize{8}{10}\texttt{:implies}\makebox[5.47cm]{}\\[-.5em]
    \fontsize{8}{10}\texttt{\{:request :uri ("/photos/", :id);}\makebox[2.5cm]{}\\[-.5em]
    \makebox[5.75cm]{\fontsize{8}{10}\texttt{:response [ :represents :photo ].\}.}}
\end{multline}

Note the automatic existential quantification of blank nodes. By turning the request also in a blank node and using the full expressive power of Notation3, we can conveniently write~\ref{eq:FullNotation3} as:
\begin{multline}\label{eq:Notation3}
    \makebox[3.44cm]{\fontsize{8}{10}\texttt{\{:photo :photoId :id.\}}}\\[-.5em]
    \fontsize{8}{10}\texttt{=>}\makebox[7.6cm]{}\\[-.5em]
    \fontsize{8}{10}\texttt{\{:request :uri ("/photos/", :id);}\makebox[2.5cm]{}\\[-.5em]
    \makebox[5.75cm]{\fontsize{8}{10}\texttt{:response [ :represents :photo ].\}.}}
\end{multline}
This minimal syntax fully reflects the functionality of the service as intended by~\ref{eq:Informal}.

\subsection{RESTdesc description format}
With the syntax and required concepts in mind, we now look at existing recommendations, proposals, and vocabularies that we can integrate to obtain an interchangeable description format.

Since RESTful services are centered around correct use of the HTTP protocol, one of the obvious elements we need is a way to describe HTTP requests. The \emph{HTTP Vocabulary in RDF}~\cite{HTTPinRDF} is already widespread, despite its W3C Working Draft status. It defines all concepts to rigorously describe HTTP messages, their structure, and their relationships.

The resource-oriented nature of RESTful services implies the use of descriptive URIs, based on a structure specific to each server. Therefore, we should be able to express the relationship between a resource and its URI. In an Internet-Draft, the IETF describes the concept of URI templates~\cite{URITemplate} to refer to a category of resources.

\needspace{2em} Below is an example of a URI template for a person in a photograph:
\begin{Verbatim}
http://example.org/photos/{photoId}/persons/{personId}
\end{Verbatim}
The identifiers between braces are variables, which can be assigned a value. For example, to get the person with identifier $3$ on photograph $241$, the URI becomes:
\begin{Verbatim}
http://example.org/photos/241/persons/3
\end{Verbatim}

While URI templates are still in draft, many implementations and applications exist. It is clear that we should include them in our design.

Finally, we need a way to tie the URI templates to HTTP request parameters such as the request URI. Also, some additional template semantics are required, for instance to describe what the response body contains. Since such a vocabulary was not available yet, we created the \emph{HTTP template} ontology\footnote{Located at {\fontsize{8}{10}\url{http://purl.org/restdesc/uri-template}}.}.

\autoref{lst:PhotoGet} shows the final description of the photo retrieval service. On a high level, we see the precondition, followed by the request and the postcondition. The concepts unique to the server itself are expressed in a \emph{server-specific ontology}. The precondition thus states that an object with a photo identifier is required. In the postcondition, we use the HTTP vocabulary to describe a \Verb!GET! request and its associated response. Finally, we use the HTTP template vocabulary to specify the URI template and the contents of the response.

Contrary to its appearance, this short description conveys a vast amount of semantic information. Of course, most importantly, there is the explicit relation expressing precisely how the input relates to the output. An alternative way to look at the implication is to state that the specified request only exists in presence of a photograph. The semantics of the quantification have been highlighted in \autoref{lst:PhotoGetQuantifiers}, which contains the same description with the explicit quantifier syntax. The incorporation of the URI template is also particularly strong: the variables in the URI have been bound to the actual values that will be present during execution. Interesting here is that these variables, due to the server-specific ontology, do not only have an \emph{associated data type}, but \emph{fully linked semantics}. For instance, if the server describes the \Verb!photoId! predicate by specifying its range as integers and its domain as photographs, this information is propagated into the URI template. \label{SemanticTemplateURI}

\begin{figure}[float=t!]
\begin{lstlisting}[caption=RESTdesc description of photo retrieval,
                   label=lst:PhotoGet, escapechar=§]
@prefix : <http://a.example.org/ontology#>.
@prefix http: <http://www.w3.org/2006/http#>.
@prefix tmpl: <http://purl.org/restdesc/http-template#>.
@prefix foaf: <http://xmlns.com/foaf/>.

{
  ?photo :photoId ?id.
}
=>
{
  _:request http:methodName "GET";
            tmpl:requestURI ("/photos/" ?photoId);
            http:resp [ tmpl:represents ?photo ].
}.
\end{lstlisting}

\begin{lstlisting}[caption=\autoref{lst:PhotoGet} with explicit quantifiers,
                   label=lst:PhotoGetQuantifiers, escapechar=§]
@prefix log: <http://www.w3.org/2000/10/swap/log#>.

§\textbf{@forAll :photo, :id.}§         # §\it $\forall$ photo, id $:$§
§\textbf{@forSome :request, :response.}§
{
  :photo :photoId :id.       # §\it photoId$($photo$, $id$)$ §
}
§\textbf{log:implies}§                  # §\it $\Rightarrow \exists$ request$, $r : resp$($request$, $r$)$§
{                            #    §\it $\land$ represents$($r$, $photo$)$ \textcolor{gray}{\rm{[\ldots]}}§
  :request §\textcolor{gray}{\rm{[\ldots]}}§ http:resp :response.
  :response tmpl:represents :photo.
}.
\end{lstlisting}

\begin{lstlisting}[caption=RESTdesc description of photo upload,
                   label=lst:PhotoUpload, escapechar=§]
{
  ?photo a foaf:Image.
}
=>
{
  _:request http:methodName "POST";
       http:requestURI "/photos";
       http:body [ tmpl:formData ("photo=" ?photo) ];
       http:resp [ tmpl:location ("/photos/" ?photoId) ].

  ?photo :photoId _:photoId.
}.
\end{lstlisting}

\begin{lstlisting}[caption=RESTdesc description of face detection,
                   label=lst:FaceDetection, escapechar=§]
{
  ?photo :photoId ?photoId.
}
=>
{
  _:request http:methodName "GET";
          tmpl:requestURI ("/photos/" ?photoId "/faces");
          http:resp [ tmpl:representsMultiple ?region ].
  
  ?region foaf:depicts [ a foaf:Person ];
          :regionId _:regionId;
          :belongsTo ?photo.
}.
\end{lstlisting}

\begin{lstlisting}[caption=RESTdesc description of face recognition,
                   label=lst:FaceRecognition, escapechar=§, belowskip=-1em]
{
  _:region foaf:depicts ?person;
           :regionId ?regionId;
           :belongsTo [:photoId ?photoId].
}
=>
{
  _:request http:methodName "GET";
            tmpl:requestURI
              ("/photos/" ?photoId "/people/" ?regionId);
            http:resp [ tmpl:represents ?person ].
  
  ?person foaf:name _:personName.
}.
\end{lstlisting}
\end{figure}

Listings~\ref{lst:PhotoUpload} to~\ref{lst:FaceRecognition} show example descriptions of other services on the same server. For photo upload (\autoref{lst:PhotoUpload}), we see the prerequisite is having an image. Note that the service description author is free to use any vocabulary, in this case the \emph{FOAF} ontology. Since the request URI is fixed, no URI template was used. The response, in contrast, will have a \Verb!Location! header with a URI containing the photo identifier. For the request, we specify the format of the \Verb!POST! body. Note how the precondition of photo retrieval (\autoref{lst:PhotoGet}) naturally follows from the postcondition of photo upload, hinting at a possible~causality.

This effect is also visible in \autoref{lst:FaceDetection} and \autoref{lst:FaceRecognition}, which both demonstrate the ease of expressing complex conditions. The required expressions involve a complicated indirection (e.g., ``the photograph contains a region that depicts a person''), yet they can be understood quite easily, while the formal semantics are sound.

When we overlook all of the above, it becomes apparent that RESTdesc descriptions are a simple and elegant way of describing Web services in an integrated semantic manner. They capture the functional aspects formally without resorting to complex artifices. The use of the HTTP vocabulary and semantic identifiers was taken from previous work~\cite{Steiner:2011p5006}, as well as the use of \mbox{Notation3} conditions~\cite{Verborgh:2010p2746}, both which were extended and combined into a single method. The resulting \mbox{RESTdesc} descriptions can be used for automatic discovery, service composition, and execution. These and other aspects will be demonstrated in \autoref{MainStory}.

\subsection{Automated interpretation and composition}
\label{Composition}
An interesting fact about Notation3 implications is that, besides the \emph{descriptive/declarative} semantics we have used so far, they also entail \emph{operational} semantics. This means that, given a reasoner that is able to make \emph{modus~ponens} inferences, the following action takes place:
\begin{equation}\label{eq:ModusPonens}
    \frac{P \Rightarrow Q, P}{Q}
\end{equation}
This is a very relevant property for RESTdesc descriptions, which \emph{enables context-based service discovery}.

For example, we might want to know what we can do on a server given we have an image. RESTdesc makes this a trivial task. \ref{eq:Image} expresses our current condition.
\begin{equation}\label{eq:Image}
    {\fontsize{8}{10}\texttt{<http://example.org/photo.jpg> a foaf:Image.}}
\end{equation}
It is also the precondition of photo upload (\autoref{lst:PhotoUpload}). Consequently, using modus ponens~\ref{eq:ModusPonens}, we can derive the postcondition of photo upload. Yet it does not stop there. The statements of the postcondition can also trigger other inferences. In the end, the result chain is:
\begin{itemize}
\item we can upload the photo, upon which it will receive an identifier
\item we can use this identifier to receive the photo
\item we can use this identifier to detect faces within it
\item we can then ask the server to recognize these faces
\end{itemize}
In addition to \emph{what} steps we can take, the inference process also tells us \emph{how} to take this steps by listing the concrete HTTP requests.

This and other examples can be verified online using the EYE Semantic Web reasoner~\cite{Euler} on the interactive RESTdesc website\footnote{
\todo{I would point here to a starting point, containing links to the reasoning examples, as well as the node server example.}
}.

An even more interesting approach is to add a goal, in a addition to starting point~\ref{eq:Image}. If we indeed want to know who is depicted in the photograph, our query might be:
\begin{equation*}\label{eq:Image}
    \fontsize{7.8}{10}\texttt{<http://example.org/photo.jpg> foaf:depicts ?person.}
\end{equation*}
The proof of the reasoner for this query, forms a list of ordered steps to obtain the desired results, again with detailed instructions on how to execute these steps. This differs from the previous output, which was just an unordered list of possible actions. Here, the result is an actual execution plan.

It is apparent that RESTdesc descriptions provide a powerful and instant way to deal with automated interpretation and composition of different web services.

% Talk about operational semantics http://www.w3.org/DesignIssues/N3Logic
% Phrase it like that the descriptions have the benefit of ALSO having operational semantics.

% Tell understanding of server-specific ontology is not necessary for composition.

% Mention we can also deal with changes over time, possibly referring to planning systems


\subsection{Compatibility and automatic translation}
\todo{Explain compatibility with CoIN, OWL-S, maybe WSDL etc., and how the translation can happen automatically.}
% Refer back to \ref{SemanticTemplateURI} for CoIN.
% Differences: RESTdesc is more flexible, RESTdesc has all HTTP methods, RESTdesc is not limited to the request URI only
% Could we automatically convert into a human-readable format?
% Content negotiation.


\section{RESTdesc Service Discovery} \label{sec:restdesc-discovery}
%%%% Responsible: THOMAS %%%%
\subsection{\todo{Theory}}
\todo{While the previous section introduced the RESTdesc description format, this section details how we can start with a single URI and learn everything about the server.}

\subsection{\todo{Google Docs story 1}}
\label{MainStory}
\todo{The details of the service composition will have been handled in \ref{Composition}.}


\section{Linking with external services}
%%%% Responsible: none yet %%%%
% Maybe we should drop this. Future separate article?
\subsection{\todo{Theory}}
\subsection{\todo{Google Docs story 2}}


\section{Adapting to change and errors}
%%%% Responsible: none yet %%%%
\subsection{\todo{Theory}}
\subsection{\todo{Google Docs story 2}}


\section{Conclusion}
%%%% Responsible: none yet %%%%

\todo{RESTdesc builds on top of RESTful principles and consists of:
\begin{itemize}
\item a semantic markup model, offering a formal description of a service's functionality, with extensive flexibility
\item an HTTP-based discovery method of services, both internal and external
\end{itemize}
}

\todo{Emphasize that RESTdesc = REST + description, e.g., that RESTful practices are a requirement.}

% Say "pragmatic" somewhere.

% Say that it is integrated, unlike OWL-S.


\section*{Acknowledgements}
\todo{Thank Niklas Lindstr\"om}

Thomas Steiner is partially funded by the EU FP7 I-SEARCH project (ref. 248296).

\todo{Thank Jos De Roo}

% back to Computer Modern for URLs in bibliography
\renewcommand{\ttdefault}{cmvtt}

\bibliographystyle{splncs03}
\bibliography{AFMS2011}

\end{document}  